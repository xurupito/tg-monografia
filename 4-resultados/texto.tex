% Capítulo 4
\chapter{Resultados}
\label{resultados}

Para cada um dos anos eleitorais analisados neste estudo são apresentados duas visualizações para o grafo de cada ano. Na primeira é possível identificar as comunidades encontradas pelo algoritmo de modularização, de maneira que cada comunidade possui uma cor diferente para seus vértices. Na segunda visualização do grafo cada vértice é colorido de acordo com a afinidade ideológica do partido.

Vértices em tons de vermelho são partidos de centro-esquerda, esquerda e extrema-esquerda. Vértices em tons de azul são partidos de centro-direita, direita e extrema-direita. Já os vértices verdes são os partidos que se declaram de centro, e os cinzas são partidos que não possuem uma classificação bem definida na literatura.

%%%%%%%%%%%%%%%%%%%%%%%%%%%%%%%%%%%%%%%%%%%%%%%%%%%%%%%%%%%
\section{\texorpdfstring{\MakeUppercase{Métricas de avaliação}}{}}
\label{resultados__metricas-avaliacao}

Como visto na sessão \ref{proposta__modelagem}, utilizamos pesos nas arestas dos grafos como forma de indicar em quantos estados dois determinados partidos fazem coligação. Assim, os grafos apresentados não indicam apenas quais são as alianças formadas, mas também o quão forte é um relacionamento entre dois partidos.

%%%%%%%%%%%%%%%%%%%%%%%%%%%%%%%%%%%%%%%%%%%%%%%%%%%%%%%%%%%
\section{\texorpdfstring{\MakeUppercase{Parâmetros para modularização}}{}}
\label{resultados__parametros-modularizacao}

Para a utilização do algoritmo de detecção de comunidades é necessário escolher um parâmetro de \emph{resolução}, quanto maior o valor deste parâmetro menos comunidades são geradas.

Assim, utilizamos o valor de resolução 1.2 para os grafos de 1994, 1998, 2002 e 2006. Com este parâmetro, o algoritmo de modularização conseguiu separar a componente gigante dos grafos em duas comunidades. Para os anos 2010 e 2014 percebeu-se a necessidade de reduzir o valor de resolução para 1.0, uma vez que ao utilizar 1.2 como parâmetro a componente gigante se tornava uma única comunidade (mais detalhes em \nameref{resultados__grafos--2010} e \ref{resultados__grafos--2014}).

%%%%%%%%%%%%%%%%%%%%%%%%%%%%%%%%%%%%%%%%%%%%%%%%%%%%%%%%%%%
\section{\texorpdfstring{\MakeUppercase{Grafos}}{}}
\label{resultados__grafos}
%% Gráficos, números, conclusões: como as comunidades ficaram, grau médio, coeficiente de clustering


\subsection{1994}
\label{resultados__grafos--1994}

\begin{figure}[H]
\center
    \subfigure[fig-1994][Comunidades identificadas]{\includegraphics[width=7.5cm]{img/grafos/1994a.png}}
    \qquad
    \subfigure[fig-1994b][Espectro político dos partidos]{\includegraphics[width=7.5cm]{img/grafos/1994c.png}}

    \caption{1994: Comunidades e espectro político}
\end{figure}

Ao comparar as duas visualizações do grafo de 1994 nota-se uma divisão evidente na afinidade ideológica predominante em cada comunidade. A comunidade identificada pela cor verde contém 11 partidos, dos quais 8 são de esquerda e 2 são de centro. A comunidade rosa apresenta 12 partidos, 6 deles são de direita e 3 de centro.

Ressalta-se que esta separação não significa que partidos da esquerda só fazem aliança com partidos de ideologia parecida por exemplo. As arestas dos grafos tendem a manter a cor dos vértices em que elas incidem, e no grafo (a) podemos ver uma grande quantidade de arestas cinzas, que apresentam esta cor por conectar vértices rosas e verdes. Isto significa que existem várias alianças sendo feitas entre os partidos das duas comunidades opostas.

Percebe-se ainda que os partidos sem classificação ideológica (\gls{PRN} e \gls{PTRB}, vértices cinzas) ficaram nas bordas do grafo, indicando que possuem grau menor em relação aos demais vértices e, portanto, fizeram menos alianças, assim como os vértices nas extremidades da comunidade verde (\gls{PCB}, \gls{PTdoB} e \gls{PRONA}).


\subsection{1998}
\label{resultados__grafos--1998}

\begin{figure}[H]
\center
    \subfigure[fig-1998][Comunidades identificadas]{\includegraphics[width=7.5cm]{img/grafos/1998a.png}}
    \qquad
    \subfigure[fig-1998b][Espectro político dos partidos]{\includegraphics[width=7.5cm]{img/grafos/1998c.png}}

    \caption{1998: Comunidades e espectro político}
\end{figure}

Percebe-se que o grafo gerado para 1998 não é conexo, contendo dois vértices sem arestas. Isso indica que existem dois partidos que não se coligaram em nenhum estado do Brasil nesse ano (\gls{PCO} e \gls{PSTU}). Estes vértices não são considerados comunidades propriamente ditas, mas optamos por manter no grafo para uma melhor visualização do comportamento geral de todos os partidos.

A divisão dos partidos de esquerda e direita nas duas comunidades manteve-se neste ano, apesar de não ser tão clara quanto em 1994. Isso se dá, em parte, ao fato de que as duas comunidades são mais densas e o grafo possui mais vértices.

É possível observar ainda três partidos de centro-esquerda e um de esquerda na comunidade com predominância de partidos de direita: \gls{PTdoB}, \gls{PGT}, \gls{PTB} e \gls{PST}.


\subsection{2002}
\label{resultados__grafos--2002}

\begin{figure}[H]
\center
    \subfigure[fig-2002][Comunidades identificadas]{\includegraphics[width=7.5cm]{img/grafos/2002a.png}}
    \qquad
    \subfigure[fig-2002b][Espectro político dos partidos]{\includegraphics[width=7.5cm]{img/grafos/2002c.png}}

    \caption{comentario}
\end{figure}

Em 2002 percebe-se a existência de uma discrepância maior no tamanho das comunidades. A menor possui 6 partidos (eram 11 em 1994). Apesar do tamanho reduzido, a comunidade ainda apresenta uma certa divisão ideológica: não existe nenhum partido de direita, sendo composta majoritariamente por partidos de esquerda.

Nota-se também que existem partidos de centro-esquerda na comunidade maior, na qual a maioria dos partidos são de direita. Este agrupamento contém novamente grande parte dos partidos centristas.

\subsection{2006}
\label{resultados__grafos--2006}

\begin{figure}[H]
\center
    \subfigure[fig-2006][Comunidades identificadas]{\includegraphics[width=7.5cm]{img/grafos/2006a.png}}
    \qquad
    \subfigure[fig-2006b][Espectro político dos partidos]{\includegraphics[width=7.5cm]{img/grafos/2006c.png}}

    \caption{comentario}
\end{figure}

Neste ano são identificadas 3 comunidades na componente principal e o tamanho das duas comunidades maiores voltou a ser mais equilibrado. O novo agrupamento - de cor cinza no grafo (a) - se posiciona longe dos demais vértices, indicando que o número de conexões destes partidos com  os demais é pequeno. Esta comunidade contém três partidos de extrema-esquerda: \gls{PCB}, \gls{PSOL} e \gls{PSTU}, e sua separação dos demais vértices nos indica que estes partidos (de ideologias muito bem definidas) costumam fazer poucas ou nenhuma aliança com os demais. Entre os três, o único a se aliar com partidos de outras ideologias é o \gls{PCB}, e nota-se que uma destas alianças é com o \gls{PSC}, partido que historicamente se opõe à ideias comunistas.

A distinção entre esquerda e direita torna-se menos presente neste grafo, a comunidade de cor verde do grafo (a) por exemplo, contém partidos de diversas afinidades ideológicas: \gls{PSDB}, \gls{PT}, \gls{PMDB}, \gls{PCdoB}, \gls{PP} e \gls{PL}, por exemplo.

\subsection{2010}
\label{resultados__grafos--2010}

\begin{figure}[H]
\center
    \subfigure[fig-2010][Comunidades identificadas]{\includegraphics[width=7.5cm]{img/grafos/2010a.png}}
    \qquad
    \subfigure[fig-2010b][Espectro político dos partidos]{\includegraphics[width=7.5cm]{img/grafos/2010c.png}}

    \caption{comentario}
\end{figure}

Em 2010 foi necessário mudar a resolução, ao utilizar 1.2 como resolução não foi possível separar a componente principal em 2 comunidades. Isso se dá, em parte, porque o grafo gerado para este ano é muito mais denso do que nos anos anteriores, como mostrado na Tabela x. Esta densidade é perceptível visualmente no grafo: os vértices estão muito próximos com uma quantidade grande de arestas entre eles.

Assim, ao aplicar o algoritmo de modularidade com resolução menor obteve-se um número maior de comunidades. Chamamos atenção para a comunidade de cor roxa que conecta apenas dois vértices: \gls{PSDB} e \gls{DEM}. A separação de uma comunidade com apenas estes dois partidos ocorreu pois o grau da aresta que os conecta é 15, indicando que esta aliança ocorreu em 15 estados brasileiros, sendo assim o segundo maior peso do grafo, atrás apenas da aliança \gls{PSTU} e \gls{PSOL}, que ocorreu em 18 estados.

O aumento de comunidades e densidade no grafo dificulta a identificação das ideologias predominantes em cada agrupamento, o que evidencia a redução (em relação aos anos anteriores) de afinidade ideológica presente nas coligações formadas em 2010.



\subsection{2014}
\label{resultados__grafos--2014}

\begin{figure}[H]
\center
    \subfigure[fig-2014][Comunidades identificadas]{\includegraphics[width=7.5cm]{img/grafos/2014a.png}}
    \qquad
    \subfigure[fig-2014b][Espectro político dos partidos]{\includegraphics[width=7.5cm]{img/grafos/2014c.png}}

    \caption{comentario}
\end{figure}

Para este grafo também foi necessário utilizar o valor 1.0 como resolução, implicando na identificação de três comunidades. Este grafo é ainda mais denso que os anteriores, e isso é perceptível visualmente. Novamente uma comunidade com partidos de extrema-esquerda é formada, sendo assim o único agrupamento de vértices distantes do restante do grafo.

No grafo (b) percebe-se que de forma geral os vértices vermelhos e azuis se encontram em regiões opostas, dando a impressão de que existe um tipo de separação por ideologia, mas isso não reflete necessariamente a realidade. O aumento na densidade do grafo e a necessidade de alterar o valor de resolução do algoritmo  de modularidade (e ainda assim não sendo possível separar o grafo em duas comunidades) indicam um aumento no número de alianças entre partidos de ideologias diferentes.

%%%%%%%%%%%%%%%%%%%%%%%%%%%%%%%%%%%%%%%%%%%%%%%%%%%%%%%%%%%
\section{\texorpdfstring{\MakeUppercase{Gráficos Gerais}}{}}
\label{resultados__graficos-gerais}

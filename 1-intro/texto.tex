% Capítulo 1
\chapter{Introdução}
\label{introducao}
\pagenumbering{arabic}
\setcounter{page}{7} % número da página, sem contar a capa.

%=====================================================

% A introdução geral do documento pode ser apresentada através das seguintes seções: Desafio, Motivação, Proposta, Contribuição e Organização do documento (especificando o que será tratado em cada um dos capítulos). O Capítulo 1 não contém subseções\footnote{Ver o Capítulo \ref{cap-exemplos} para comentários e exemplos de subseções.}.

\begin{comment}Por outro lado, existem também o cenário em que partidos pequenos são ideologicamente consistentes e se unem a partidos com ideais e objetivos em comum. \end{comment}

A formação de alianças é importante para os partidos políticos brasileiros. Em um sistema altamente fragmentado, unir forças para atingir objetivos eleitorais em comum é uma das formas de alguns partidos conseguirem se destacar e obter resultados durante as eleições.

Quando partidos pequenos não possuem representatividade suficiente para disputar um cargo no Executivo, por exemplo, é provável que estes acabem se aliando a um partido maior, em troca de vantagens como cadeiras no Legislativo, cargos em ministérios, secretarias e empresas estatais em caso de uma campanha bem sucedida. Estas alianças - as chamadas \emph{coligações partidárias} - muitas vezes ocorrem por trocas de favores, o que acaba sendo prejudicial para a política brasileira. Queremos verificar com este trabalho, utilizando o repositório de dados do TSE e visualizações em grafo, se as coligações são formadas por partidos de afinidade ideológica ou se os interesses políticos acabam se sobressaindo durante a formação das alianças. Fazendo uma análise nos anos de eleição federal compreendidos entre 1994 e 2014, utilizamos um algoritmo de modularidade para a identificação de comunidades nos grafos de coligações para o cargo de deputado estadual, onde foi observado que estas comunidades apresentavam uma distinção clara de afinidade ideológica entre os seus partidos.

Este trabalho está organizado da seguinte maneira: No capítulo \ref{conceitos} são apresentados alguns conceitos  sobre teoria de grafos, redes sociais e funcionamentos básicos da política brasileira. No capítulo \ref{proposta} são explicados o objetivo deste estudo e o método utilizado para realizar a análise dos dados afim de atingir os objetivos propostos. No capítulo \ref{resultados} são mostrados os resultados obtidos através de comparativos entre as comunidades identificadas nos grafos e o eixo político dos partidos presentes nas mesmas. Por fim, o capítulo \ref{conclusao} traz uma conclusão geral a respeito do estudo e dos resultados obtidos.


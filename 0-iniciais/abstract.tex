The present work proposes to analyze how Brazilian political parties constitute their alliances in the legislative sphere, for positions of state deputies. Through the use of the data repository obtained from the portal of the Superior Electoral Court (TSE), graphs were generated for each year of federal election, which covers the period from 1994 to 2014. This process arose with the purpose of evaluating if party coalitions follow some political-ideological pattern. Necessary concepts are presented for the understanding of the study, followed by the work proposal, the methodology used and the results achieved. It finishes with a general conclusion regarding the data obtained.



\textbf{Keywords:} social networks, graphs, politics, political parties, coalitions.

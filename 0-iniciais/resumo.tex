O presente trabalho  propõe-se a analisar como partidos políticos brasileiros constituem suas alianças na esfera legislativa, para cargos de deputados estaduais. Através da utilização do repositório de dados, obtido no portal do Tribunal Superior Eleitoral (TSE), foram elaborados grafos para cada ano de eleição federal, que, compreende o período dos anos de 1994 à 2014, tal processo surge com o intuito de se avaliar se as coligações partidárias seguem algum padrão político-ideológico. Apresentam-se conceitos necessários para o entendimento do estudo, mostra-se a proposta do trabalho, a metodologia utilizada e em seguida os resultados alcançados. Finaliza-se com uma conclusão geral a respeito dos dados obtidos.



\textbf{Palavras chave:} redes sociais, grafos, política, partidos políticos, coligações.

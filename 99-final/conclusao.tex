% Capítulo 2
\chapter{Conclusão e trabalhos futuros}
\label{conclusao}

Utilizando os conhecimentos em computação adquiridos durante a graduação, especialmente em teoria de grafos e redes sociais, foi possível analisar como se comportaram as coligações partidárias no Brasil no período entre 1994 e 2014. Constatou-se que embora os partidos formassem um número maior de alianças com aqueles de ideologias políticas similares, esta distinção tornou-se menos evidente nos anos subsequentes. O número de alianças entre diferentes legendas cresceu consideravelmente (eram 142 em 1994, alcançando 356 em 2014), enquanto a quantidade de partidos no país não teve uma mudança significativa. Ao observar o grau médio ponderado dos grafos, percebeu-se que cada partido fazia em média, 33,043 alianças em 1994, número que subiu para 69,65 no ano de 2014. Constatou-se ainda que a partir de 2006 não foi possível separar os grafos em duas componentes mantendo o valor de resolução utilizado nos anos anteriores, e mesmo ao reduzir este parâmetro, a quantidade mínima de \emph{clusters} identificados subiu para 3.

Um dos maiores desafios encontrados durante o desenvolvimento deste trabalho foi utilizar o repositório no site do \gls{TSE}. Arquivos com formatações inconsistentes, conteúdos incompletos e uma carência por filtros e modelagem adequada dos dados dificultam a utilização de informações que são fundamentais para estudos sobre a política brasileira. Com isso, esperamos em trabalhos futuros conseguir elaborar um meio mais eficiente de utilizar os dados do repositório, através da criação de uma modelagem relacional e uma \emph{\gls{API}}, podendo assim,  viabilizar e incentivar mais estudos a cerca destes dados e fornecer uma maneira simples do público em geral ter conhecimento sobre os mesmos.




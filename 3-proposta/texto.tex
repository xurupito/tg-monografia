% Capítulo 3
\chapter{Proposta}
\label{cap3_proposta}

%%%%%%%%%%%%%%%%%%%%%%%%%%%%%%%%%%%%%%%%%%%%%%%%%%%%%%%%%%%
\section{\texorpdfstring{\MakeUppercase{Metodologia}}{}}
\label{secao_metodologia}

O site do \gls{TSE} fornece um repositório de dados eleitorais\footnote{http://www.tse.jus.br/eleitor-e-eleicoes/estatisticas/repositorio-de-dados-eleitorais-1} que contém um compilado de informações brutas das eleições no Brasil, sendo assim um meio para pesquisadores, imprensa e pessoas interessadas analisarem dados de candidatura, eleitorado, resultados e prestação de contas. Estes dados são fornecidos em arquivos no formato \emph{.csv}, de forma que consultas, filtros e cruzamento de dados são de responsabilidade do pesquisador.

Além dos arquivos, o site do \gls{TSE} fornece uma documentação que visa explicar como os dados estão dispostos em cada tipo de arquivo. Após estudo da documentação fornecida, elaboramos um modelo relacional dos dados para possibilitar um melhor entendimento do conteúdo disponível e decidir quais informações seriam interessantes para análise. Nesta etapa percebeu-se que existiam arquivos com dados incompletos e formatações diferentes na disposição de seus conteúdos, gerando assim uma redução de dados que poderiam ser analisados forma satisfatória.

Esta inconsistência encontrada em alguns dados nos levou a desistir de uma ideia inicial, na qual pretendia-se analisar as coligações para eleições presidenciais. Dessa forma, optou-se por focar o estudo nas coligações formadas para disputa ao cargo de deputado estadual.

Dentre os arquivos estudados, avaliou-se que os dados referentes aos candidatos eram os mais completos, apresentando inclusive informações sobre partidos e coligações. Dessa forma, optamos por observar o relacionamento entre os partidos políticos ao longo dos anos de eleição.


Passos tomados para alcançar os objetivos
\todo{explicar o parser}

%%%%%%%%%%%%%%%%%%%%%%%%%%%%%%%%%%%%%%%%%%%%%%%%%%%%%%%%%%%
\section{\texorpdfstring{\MakeUppercase{Modelagem}}{}}
\label{subsecao_modelagem}

Após a obtenção dos dados formatados da maneira desejada, foi definido que seriam analisadas coligações partidárias estaduais em âmbito nacional, para o cargo de deputado. Duas possíveis abordagens para a modelagem em grafos surgiram:
\begin{itemize}
    \item Gerar um grafo por estado. Cada partido político seria um vértice e os partidos que participassem de uma mesma coligação seriam conectados por uma aresta.
    \item Gerar um grafo ponderado por ano. Cada vértice seria um partido e as arestas conectariam partidos de mesmas coligações. O peso das arestas representaria em quantos estados os dois partidos participam de uma mesma coligação naquele ano.
\end{itemize}

\todo{talvez seja possível citar que analisar grafos por cada estado demandaria uma quantidade maior de grafos e nos traria resultados menores e mais específicos}

Avaliou-se exemplos de grafos dentro do primeiro cenário, e percebeu-se que em cada estado o grafo correspondente apresentava diversas cliques, cada uma sendo uma coligação.

\todo{imagem do grafo do paraná}

Percebemos que esta abordagem não seria proveitosa, pois exigiria uma grande quantidade de grafos e mesmo assim não seria possível obter as informações desejadas para este trabalho. Desta forma, adotamos a segunda abordagem.

%%%%%%%%%%%%%%%%%%%%%%%%%%%%%%%%%%%%%%%%%%%%%%%%%%%%%%%%%%%
\section{\texorpdfstring{\MakeUppercase{Restrições}}{}}
\label{secao_objetivo_geral}

\todo{colocar restrições encontradas, falar do espectro político}

Definição de Auto-denominação dos Partidos: (falar que nem todos se auto declaram, mas de certa forma esse seria um dos objetivo deste trabalho. Explicar que os partidos grandes tem a auto declaração)
s


Por fim, entendemos que foge da proposta deste trabalho discorrer sobre questões socio-políticas a respeito dos dados apresentados, uma vez que estas questões podem ser melhor discutidas por estudiosos de outras áreas do conhecimento, como sociólogos, historiadores e cientistas políticos.

%%%%%%%%%%%%%%%%%%%%%%%%%%%%%%%%%%%%%%%%%%%%%%%%%%%%%%%%%%%
\section{\texorpdfstring{\MakeUppercase{Revisão da Bibliografia}}{}}
\label{secao_revisao_bibliografia}

Algo light. Algum trabalho que possa ter motivado o nosso


%%%%%

%%%%%%%%%%%%%%%%%%%%%%%%%%%%%%%%%%%%%%%%%%%%%%%%%%%%%%%%%%%
\section{\texorpdfstring{\MakeUppercase{Objetivo Geral}}{}}
\label{secao_objetivo_geral}

Este trabalho propõe-se a utilizar o repositório de dados do \gls{TSE} para a construção de grafos referentes às informações de coligações partidárias no Brasil nos anos 1994 à 2014, com o objetivo de exibir métricas e visualizações que possam indicar se as coligações são formadas por partidos de ideologias semelhantes.

%%%%%%%%%%%%%%%%%%%%%%%%%%%%%%%%%%%%%%%%%%%%%%%%%%%%%%%%%%%
\section{\texorpdfstring{\MakeUppercase{Objetivos Específicos}}{}}
\label{secao_objetivos_especificos}

%% Objetivos Específicos: ("pontual", em formato de lista. São pequenos objetivos para chegar/alcançar o objetivo geral do trabalho)

\subsection{Modularização}
\label{subsecao_modularizacao}

Modularizar em poucas comunidades (porque poucas?)

%%%%%
\subsection{Informações apresentadas}
\label{subsecao_infos_apresentadas}

qual é o interesse em apresentar as informações sobre grau médio, coeficiente de clustering, ...
